\documentclass[12pt, letterpaper]{article} 	% use "amsart" instead of "article" for AMSLaTeX format
\usepackage[top=1in, bottom=1in, left=1in, right=1in]{geometry}    		% See geometry.pdf to learn the layout options. There are lots.
	
\geometry{letterpaper}
\setlength{\parindent}{0pt}
\setlength{\voffset}{0in}  
%\setlength{\headsep}{0pt}
\setlength{\footskip}{0.75in}  		 
%\geometry{landscape}    		% Activate for rotated page geometry
%\usepackage[parfill]{parskip} 		% Activate to begin paragraphs with an empty line rather than an indent
\usepackage{graphicx}
\usepackage[skipabove = 0.2in, skipbelow = 0.2in]{mdframed}
	\mdfdefinestyle{inception}{skipabove = -0.1in, skipbelow = 0.1in, , innertopmargin=-0.2in}
\usepackage{times}	
\usepackage{amsmath,amssymb}
\usepackage{mathtools}
\usepackage{accents}
\usepackage{setspace}
\usepackage{breqn}
\usepackage[makeroom]{cancel}
\usepackage{booktabs}
\usepackage[dvipsnames]{xcolor}
\usepackage{empheq}
\usepackage[unicode=true, colorlinks=true, linkcolor=blue, citecolor=blue, urlcolor=blue]{hyperref}
\urlstyle{same}
\usepackage[english]{cleveref}
\usepackage[utf8]{inputenc}
\usepackage{verbatim}
\usepackage{multicol}
\usepackage{enumitem}
	\setlist{nosep}
\usepackage{scrextend}
\usepackage[none]{hyphenat}
\usepackage{changes}
\definecolor{umichBlue}{cmyk}{1, .6 , 0, .6}
\definecolor{umichMaize}{cmyk}{0, .18, 1, 0}

\usepackage[super,numbers,compress]{natbib}
\bibliographystyle{apalike}

\newlength{\dhatheight}
\newcommand{\dhat}[1]{%
	\settoheight{\dhatheight}{\ensuremath{\hat{#1}}}%
	\addtolength{\dhatheight}{-0.2ex}%
	\hat{\vphantom{\rule{1pt}{\dhatheight}}%
	\smash{\hat{#1}}}}


\setlength\fboxrule{0.08em}
\newcommand*{\boxedcolor}{umichBlue}
\makeatletter
\renewcommand{\boxed}[1]{\textcolor{\boxedcolor}{%		%if I box things, this turns the box to blue
  \fbox{\normalcolor\m@th$\displaystyle#1$}}}
\makeatother

\newcommand\numberthis{\addtocounter{equation}{1}\tag{\theequation}}

%\DeclareMathSizes{12}{14}{10}{10}

\linespread{1.2}
\newcommand*{\Resize}[2]{\resizebox{#1}{!}{$#2$}}

\newcommand{\twqq}[0]{\qquad \qquad}
\newcommand{\thqq}[0]{\qquad \qquad \qquad}
\newcommand{\fqq}[0]{\qquad \qquad \qquad \qquad}
\newcommand{\nq}[0]{\mkern-18mu}
\newcommand{\nqq}[0]{\mkern-36mu}
\newcommand{\ntwqq}[0]{\mkern-72mu}
\newcommand{\tbf}[1]{\textbf{#1}}
\newcommand{\mbf}[1]{\mathbf{#1}}

\title{\Huge GitHub Notes}
\author{Ellen Mulvihill \\ Created 6/29/18}
%\date{}
%\affiliation{University of Michigan}
%SetFonts							
\begin{document}
\maketitle
\tableofcontents
\clearpage

%%%%%%%%%%%%%%%%
\section{Introduction}
%%%%%%%%%%%%%%%%

%%%%%%%%%%%%%%%%%
\section{Getting Started}
%%%%%%%%%%%%%%%%%

%%%%%%%%%%%%%%%%%%%%%%%%
\subsection{Downloading and Installing git}
%%%%%%%%%%%%%%%%%%%%%%%%
\begin{itemize}
	\item First, check if you have git by typing git --version into the command line. It should give back the version of git installed. If not, you probably don?t have git.
	\item If you don?t have git, there are a few ways to install it:
	\begin{itemize}
		\item On a Mac, it should prompt you to install Xcode Command Line Tools after running {\tt git --version }unsuccessfully
		\item On a Linux, you can do{\tt\, sudo dnf install git-all} or \\ {\tt sudo apt install git-all}, depending if you're on Fedora or a Debian-based distribution, respectively
		\item Download from the website: \url{https://git-scm.com/}
		\item From the source, follow instructions here: \\ \url{https://git-scm.com/book/en/v2/Getting-Started-Installing-Git }
	\end{itemize}
\end{itemize}

%%%%%%%%%%%%%%%
\subsection{Short Version}
%%%%%%%%%%%%%%%

%%%%%%%%%%%%%%%
\subsection{Long Version}
%%%%%%%%%%%%%%%


%%%%%%%%%%%%%
\section{Notes}
%%%%%%%%%%%%%

%%%%%%%%%%%%%%
\subsection{Resources}
%%%%%%%%%%%%%%
The Pro Git book by Scott Chacon and Ben Straub\cite{ProGit}: \url{https://git-scm.com/book/en/v2}.


%%%%%%%%%%%%%%%%%%%%%%
\subsection{List of Common Commands}
%%%%%%%%%%%%%%%%%%%%%%

%%%%%%%%%%%%%%%%
\subsection{Merge Conflicts}
%%%%%%%%%%%%%%%%

%%%%%%%%%%%%%%
\subsection{Branching}
%%%%%%%%%%%%%%

%%%%%%%%%%%%%%%%%%%%%%%%
\subsection{Forking: Copying Public Code}
%%%%%%%%%%%%%%%%%%%%%%%%

%%%%%%%%%%%%%%%%%%%%
\subsection{List of Common Errors}
%%%%%%%%%%%%%%%%%%%%

\subsubsection{Detached Head}

\clearpage
\bibliography{GitHubNotes}
\addcontentsline{toc}{section}{References}

\end{document}
